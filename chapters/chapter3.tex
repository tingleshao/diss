\chapter{BLE}


In this modern age of the Internet of Things (IoT), it is now possible to literally glue tiny computers to everyday objects, so that they can sense, react, and tell their own stories. The IoT community has embraced wireless standards such as Bluetooth Low Energy (BLE) and developed programmable `beacon' devices that periodically broadcast a small amount of preloaded data, while lasting for multiple years on a coin-cell battery. Broadcast messages from beacon devices typically contain information about an object, a location, a web-resource, or just an arbitrary string. This connectionless mode of BLE does not require a receiver to pair/bond or connect to a sender, and hence, there is no overhead of connection setup and no inconvenience of requiring a user to enter pins and passwords. These broadcast messages are received by a BLE capable mobile device to obtain relevant information just-in-time and on-the-spot.  Emerging applications of beacon devices include advertising merchandise in retail stores, identifying late passengers at the airports, authorizing people at the hospitals, smarter signage, indoor navigation, and tracking moving platforms like airline cargo containers, computers on wheels, museum artworks, or even humans.


The enabling technology behind these applications is the ability of a beacon to simply broadcast a few bytes of data (usually a URL or a UUID) as BLE 4.0 advertisement packets. The bound in data rate comes from the lifetime requirement of these devices. Such a tight budget on payload size and the maximum data rate have limited a beacon's capability to only be able to broadcast an identifier or a small amount of text (effectively $\sim$30 bytes). The next generation BLE 5.0 beacon is expected to have an 8X increase in broadcasting capacity ($\sim$256 bytes). Such an increase opens up the possibility to design beacons that can serve larger assets, e.g., an image, carried by connectionless BLE advertisement packets. However, even a simple $72\times 72$ PNG image, such as the Android launcher icon, has a size of over 3KB. To store and broadcast this image, either we require to use a dozen of BLE 5.0 beacons, or we will have to accept a very long image transmission and loading time.

Image compression is a natural way to deal with this problem. Existing image compression algorithms, however, fail to achieve the desired compression ratio for an image to be broadcasted over BLE. Hence, a fundamental challenge toward realizing an image beacon is to devise an algorithm that efficiently represents an image using as few bits as possible, while taking into account the application-driven limits on the number of usable beacons per image, broadcast message size, data rate, latency, and lifetime. In an earlier work~\cite{shaoyears}, we devised an image beacon system that broadcasts binary images of a few limited categories (e.g., handwritten characters) only. This paper is a continuation to that line of work, but this time, we have taken a harder challenge, i.e., to develop a beacon system that works for color images, e.g., images taken with a mobile phone.


%Emerging applications of beacon devices include advertising merchandise in retail stores~\cite{pierdicca2015low}, identifying late passengers at the airports, authorizing people at the hospitals, smarter signage, indoor navigation~\cite{martin2014ibeacon}, and tracking moving platforms like airline cargo containers, computers on wheels, museum artworks, or even humans~\cite{conte2014bluesentinel}. The enabling technology behind these applications is the ability of a beacon device to simply broadcast a few bytes of data (called UUID) as BLE advertisement packets at a rate of less than 16 bytes/sec. The bound in data rate comes from the lifetime requirement of these devices. Such tight budgets on payload and maximum data rate has limited a beacon device's capability to only be able to broadcast an identifier or a small amount of text (about 16--18 bytes). To transmit a moderate sized image, either we require to use hundreds of beacon devices, or we will have to accept a very long transmission delay.

%Hypothetically, if we could broadcast high-resolution images from a beacon device in real-time, the technology would enable even more powerful and feature rich applications. Like the web has evolved from serving hypertexts to streaming multimedia contents, we envision that the natural successor of a beacon device would be the one that broadcasts images, while meeting the same energy and lifetime requirement. Applications of such an image beacon system would be in scenarios where there is no Internet connectivity but there is a need for storing and broadcasting information that can be best described by an image. Potential applications of beacon image systems include coordinating rescue workers in disaster areas, creating a bread-crumb system for adventurous hikers and mountaineers, remote surveillance (when coupled with a camera), or even a simple system just to let someone know that `I was here'.


Being able to broadcast images from beacons enables more powerful and feature rich applications than the ones supported by today's beacons. We envision that like the web has evolved from serving hypertexts to streaming multimedia contents, the natural successor of today's beacon devices would be the ones that broadcast images. Applications of image beacons would be in scenarios where there is no Internet connectivity but there is a need for storing and broadcasting information that can be best described by an image. For example, coordinating rescue workers in disaster areas, creating a bread-crumb system for adventurous hikers and mountaineers, remote surveillance (when coupled with a camera), or even a simple system just to let someone know that `We were here'.  Recently, Google started to experiment with an idea called \textit{`Fat Beacons'}, where they are looking into broadcasting html pages over BLE. However, for lack of a suitable image compression technique, the pages do not support images.

%Our work will complement such efforts.

%In this paper, we chase this seemingly impossible goal of creating a beacon device that efficiently broadcasts images over a long period. As a first step toward realizing an image beacon, we explore the challenges to broadcasting binary images of different categories (e.g., alpha-numeric characters, basic shapes, and arbitrary binary images), and design algorithms to efficiently store contents of an image inside a set of beacon devices. The set of beacons simultaneously broadcasts chunks of an image over BLE, which are captured by a mobile device to reconstruct the image. A fundamental challenge toward achieving this is to efficiently represent an image using as few bits as possible. Standard image compression algorithms are not good enough to archive the required compression ratio so that an image can be stored inside a beacon. We investigate image approximation/coding techniques that take into account the limits on number of beacon devices, number of bits available in a beacon device, data rate, latency, and lifetime. Based on empirical analysis, we devise a patch-based image approximation algorithm which greatly reduces the image data while keeping the image distortion under a threshold. We investigate the tradeoffs between the image quality and the power consumption to determine the best set of parameters for the system under user-specified constraints.

In this paper, we chase this seemingly impossible goal of creating an image beacon system that efficiently broadcasts color images, carried by BLE broadcast messages, over an extended period of time. We propose a self-contained system that stores and broadcasts actual image contents as opposed to IDs, links, or URLs of an image. We assume availability of no additional information on the broadcasted image from any other sources -- globally (on the web) or locally (on a user's smartphone that receives the broadcast).

The crux of the system is an algorithm that analyzes an image to identify its `important' semantic regions (as defined by the user or the use case) and then encodes them differently than the rest of the image to reduce the overall image size. The image data are written to and read from the image beacon system using a smartphone application, which runs the proposed compression and rendering algorithms. We use the term `beacon system' instead of `a beacon', since a compressed image may still require more than one physical beacons to ensure its acceptable quality. Allowing multiple beacons per image makes the system flexible. It widens our scope for optimizations and helps satisfy users who are willing to dedicate more beacons for better results. Besides, until BLE 5.0 is available, we need to simulate its broadcast capacity with multiple BLE 4.0 devices anyways.

%a standard JPEG image, converts it to binary format, and shows the user a preview of the compressed image to be written. The user is allowed to change the settings (e.g., the number of available beacons and/or expected device lifetime) and the app immediately shows the best possible compressed image under these constraints.

We have developed a prototype of an image beacon system using a set of commercially available Estimote beacons~\cite{ESTIMOTE}, and developed an Android application that takes images of an object of interest along with user-specified requirements and constraints on broadcasting the image as inputs, generates previews of the image to be written, writes the image representation into a set of beacons, and reads the broadcasted image back. Figure~\ref{fig:beacons} shows an example scenario where a user snaps photos of a gnome statue which he is interested in broadcasting. The smartphone application performs image processing on the phone to produce multiple versions of broadcast image. The user selects one of these compressed images that satisfies his requirements (e.g. available beacons, image quality, lifetime, and image loading latency). The user is allowed to change his requirements and the app immediately shows options for the best possible compressed images under those constraints. The application writes the image data into the beacon system and the image is broadcasted by the beacons. A reader application reads the broadcasted image and displays it on the phone.


We perform an in-depth evaluation of the beacon system. We describe a set of results showing the tradeoffs between system lifetime and image quality, when the image type and the number of beacons are varied. We also deploy an image beacon system indoors, and perform a user study in a real-world scenario in order to have a subjective measure of the quality of the received images, where a group of $20$ participants are asked to identify objects from their beaconed images of various resolutions, and locate it among a set of similar looking objects in the real-world.

The main contributions of this paper are as follows:

\begin{enumerate}%[leftmargin=10pt]
	%\vspace{-0.25em}
	\item To the best of our knowledge, we are the first to propose an image beacon system that uses multiple BLE beacons to broadcast color images over the BLE advertisement messages.
	%\vspace{-0.25em}
	\item We have devised an image approximation algorithm that is tailored to the need of an image beacon system. We quantify the tradeoffs between the image quality and the device lifetime, and determine the best set of parameters, under the user-specified constraints on the number of beacons, latency, and expected system lifetime.
  \item We have developed and evaluated a prototype of an image beacon system that broadcasts color images of various types (e.g., near-distance indoor and outdoor objects, road signs, and buildings). Our evaluation shows that one BLE 5.0 beacon would be capable of broadcasting good-quality images (70\% structurally similar to original images) for a year-long continuous broadcasting, and both the lifetime and the image quality improve when more beacons are used.
	%}
\end{enumerate}
